\section{Timo Babst}

In the first week, I looked for a suitable Translation API. After trying out some others, I chose Glosbe and contacted the creators to make sure
that we would not get blocked for sending too many translation reqeusts. After that, in the following weeks, I implemented the Translator with Joris.
We faced a couple of issues since the Glosbe API is not documented. Moreover, we tried to formualte the GET request for the API using
Play's built-in functions, which was very tedious. In the end, we decided to use Apache's framework for the request.
During the last weeks, I implemented the Keyword Selection page, which prompts the user to enter his keywords, choose his tranlsation languages and remove
inaccurate translations if need be.




