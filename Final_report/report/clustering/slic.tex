\section{ SLIC algorithm}
\label{sec:K-M_clustering}

We implemented this algorithm, but did not integrated it to the final version of the project since the result was not really satisfactory. Adding it to the display shouldn't be hard though, since everything is already implemented for it. 

The SLIC (Simple Linear Iterative Clustering)  algorithm \cite{SLIC1}, published in June 2010, introduce a novel algorithm that clusters "pixels" in the combined n-dimensional density and image plane space to efficiently generate compact, nearly uniform superpixels. Due to its simplicity and its efficiency the algorithm quickly became very popular and often use in object class recognition, medical image segmentation and many other.\\

The SLIC algoithm is basically an adaptation of K-means for superpixel generation, with two important distinctions:

\begin{itemize}

\item The number of distance calculations in the optimization is dramatically reduced by limiting the search space to a region proportional to the superpixel size. This reduces the complexity to be linear in the number of pixels N—and independent of the number of superpixels k.

\item A weighted distance measure combines density and spatial proximity while simultaneously providing control over the size and compactness of the superpixels.

\end{itemize}

More information on the SLIC algorithm can be found in \cite{SLIC1} (original paper) and \cite{SLIC2} (State of the art of SuperPixel techniques).\\ 

The algorithm takes as input the segmented map image and, in order to produce roughly equally sized homogene region, distribute $K$ cluster centers $C_k, k \in \{1, \ldots, K \}$ uniformly over $I$. Let $S = \sqrt{K}$, It is done by over-sample $I$, create $S \times S$ "superpixel", place $C_k$ at the center of superpixel (figure 2) and compute the average density of tweet in the superpixel. 

  \begin{figure}[H]
 \centering
 \includegraphics[scale=0.4]{\raccg{fig2.png}} 
 \end{figure}
 
 We then run a local K-mean using the following distance:
 
 
$D_s(i,j) = \sqrt{(x_i - x_j)^2 +(y_i - y_j)^2 }$ where $x_i$ and $y_i$ refer to the spatial location of pixel $i$ and $j$ \\

$D_t(i,j) = \vert (d_i - d_j) \vert $ where $d_i$ refer to the density of tweet in pixel $i$ \\

$D(i,j) =\sqrt{D_t^2 + (\frac{D_s}{S})^2 \times \gamma}$\\

The free parameter of the algorithm are $K$, $\gamma$ and $\textit{ Low threshold}$.  The parameter $K$ impact the granularity of clustering, it should be $\geq 10$,  $\gamma$ is a parameter that allow a trade off between homogeneity in space and homogeneity in density of tweet of clusters. Finally $\textit{Low threshold}$ set the number of iteration that will be necessary for the algorithm to converge.\\

More detailed and complete adaptation of SLIC to our project can be found \href{https://documents.epfl.ch/users/g/go/gouedard/public/KmeanClustering.pdf}{\underline{here}}.