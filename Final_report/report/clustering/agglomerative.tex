\section{ Agglomerative clustering}
\label{sec:agg_clustering}
In order to be able to extract as much information as we could from the gathered data, we decided to implement a clustering algorithm. Lewis and Adrien consulted some literature on the subject, and decided to devise a hierarchical clustering algorithm in order to have different levels of clustering to display on the map according to the zoom. 

The challenging part was to find a good metric for clustering since, at that point, we had very little statistics about the data that we were able to gather. We tried to find some general relation that would adapt to any reasonable set of results. This algorithm is called "HomeCooked" clustering algorithm.

The algorithm proceeds as follows: 
The GeoSquares (which are the smallest geographical unit of Tweet collection) are abstracted away and replaced by ``leafClusters'', that is, a simple square on an n x m grid. A cluster is a rectangle formed by contiguous leafClusters which are ordered according to the position of their center on the grid. They do not interleave. 

At the very first level, we have the raw data (unclustered results, the basic GeoSquares). The higher we get, the more clustered the data becomes. 

Each level results from applying the clustering algorithm to the previous level. 

At each step, for each cluster, we look at the list of clusters that are after the current one in the ordered list, and try to find the one such that the area formed by aggregating them (that is, if we denote by $i$ and $j$ the clusters to be merged, the rectangle with bottom-left corner coordinates $(min_{k\in \{i,j\} } x_k, \min_{k\in \{i,j\}} y_k )$ and top-right corner coordinates $(max_{k\in \{i,j\}} x_k, \max_{k\in \{i,j\} } y_k)$ ) is $\leq \alpha \times b \times \textit{(area of the whole region)}$ and the tweet density is maximal and superior to $(1-b)\times \frac{\textit{(total number of tweets)}}{\textit{(total area)}}$, where $b$ corresponds to the level at which we are clustering and $\alpha$ is an application specific corrector constant. 

The intuitive understanding behind this algorithm is that, at each level, we consider only regions of reasonable area in order to simulate proximity between areas. The higher the level, the greater distance we're allowed to aggregate. This generates a hierarchy since near clusters are clustered first and then, in the following steps we are able to cluster further ones. Another way to see that is the following: at low levels we allow only numerous small clusters, while in higher levels we have less clusters, but they are bigger. The second condition is there to ensure that we cluster elements only when their joint density is significant, with respect to the level at which we are. For low levels, we require that the region have only a small percentage of the overall density. The higher we get in the clustering level, the bigger this percentage has to be. As a result we get a nice hierarchy of clusters, where each cluster is formed only when the area considered seems interesting in terms of density of tweets. 

Once the interface with the application was ready, we were able to test our algorithm and were satisfied with the results. 

Pierre verified the theory behind our HomeCooked algorithm and proposed another one, based on a K-means algorithm called SLIC (see next section).

Our HomeCooked algorithm was finally kept and integrated into the application. At some point, we discovered that the implementation was a bottleneck for the application when we increased the number of initial GeoSquares. Luckily, Adrien was able to refactor it so that it is now more efficient. He also rewrote it in a Map-Reduce form so that, if ever needed, the addition of a Spark or Hadoop implementation could be easily done. 

Finally, for the display, we decided to approximate a cluster by its center and a disk whose diameter is computed so that the area of the disk corresponds to the area of the original rectangle corresponding to the cluster. The opacity of the disk is computed according to the density of the cluster. We also added a slider on the View so that we can select the level of clustering to display. 
