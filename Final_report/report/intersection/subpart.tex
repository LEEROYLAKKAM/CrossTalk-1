\section{Partitioning the map}
\label{sec:partitioning}

Adrien and Lewis designed and implemented a scheme for partitioning queries to
the Twitter API over the geographical area that the end-user selected. We could
interact with the API by specifying individual squares in which to look for
Tweets. The initial design attempted to perform this task dynamically, using
feedback from the Tweets already received to increase the number of queries
performed in certain areas. Thus, if there appeared to be many Tweets in some
place, more queries would be spawned to increase the granularity of the
sampling in that area.

This scheme aimed to reduce the number of requests made to the Twitter API.
However it created extra work for the server, which would need to deduplicate
Tweets and those Tweets would take longer to collect. Instead, since the number
of API requests didn't seem to be the limiting factor, we spawned requests
at the maximum granularity right from the start.

Since we were using the Play framework, the implementation was performed in
Akka. We created a GeoPartitioner actor which would perform the appropriate
splitting for each user-selected square and start the queries. We initially had
a separate Counter actor which would count Tweets corresponding to a query.
However Akka actors are actually very expensive and creating one counter for
every query created a large amount of unnecessary overhead. In the final
version these counters were moved into the GeoPartitioner itself.

Finally we had to decide on an interface that would be used to retrieve results
from the GeoPartitioner. The results would be displayed using squares of
varying opacity to distinguish between areas with different amounts of Tweets,
and it should be possible to get the counts of Tweets in each subarea, possibly
filtering out those not contained in a specified place. We simply created a
message for each of these tasks, to which the GeoPartitioner responds with the
desired data.

\section{Interaction between the front and back end}
\label{sec:task_controller}

Joris and Lewis designed and implemented the actions which are called by the
front end once the user has selected the search parameters.  These were
implemented using Play \verb+Action+s which are invoked upon HTTP requests to
particular routes.

There are a handful of these actions, the most important of which are for
starting the data collection and computing the display data.  When the
\verb+Start+ action is invoked the \verb+GeoPartitioner+ actors are
instantiated and started with the desired parameters.  In order to be able to
display intersections between the requested keywords, we spawned three
partitioners for each user-selected area: one for each of the keywords, and
another for their intersection.  Twitter's search API allows keywords to be
combined with logical conjunctions and disjunctions, which we used to create
these queries.

When it comes to displaying the data, there are two requests the user can make:
they can request either clustered data or squares with opacities.  In both
cases, the Tweet counts for the Venn diagram are assembled and filtered to
whatever square is currently displayed.  The desired data (either clusters or
opacities) are collected and transformed into the format expected
by the front end, and everything is passed to a view which is shown to the
end-user.  There is also an action to refresh the Venn diagram to display the
statistics for the area currently visible to the end-user.

In order to test these actions, it is helpful to be able remove the dependency
on the Twitter API.  By doing this, it is possible to write unit tests which
only run the task controller code.  This reduces the number of reasons for which
the tests can fail, making them more reliable and useful.  To accomplish this,
we proceeded as is done in the Cake pattern.  All of the actors that were used
were put into an abstract trait, \verb+actors+.  There was then a concrete
implementation of this trait which used the real actors, contacting the Twitter
API and so on, and a test implementation which mocked those actors instead.
